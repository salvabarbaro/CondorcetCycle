\documentclass[a4paper, 12pt]{scrartcl}
%
\usepackage[utf8]{inputenc}
\usepackage{lmodern}
\usepackage{xcolor}
\usepackage{enumitem}
\usepackage{amsmath}
\usepackage{amsfonts}
\usepackage{amssymb}
\usepackage{mathrsfs}
\usepackage{graphicx}
\usepackage[nice]{nicefrac}
\usepackage{natbib}
\usepackage{ntheorem}
\usepackage{authblk}
\usepackage[breaklinks]{hyperref}
\hypersetup{colorlinks, 
	linkcolor=blue, 
	citecolor=blue,
	urlcolor=blue}
\newenvironment{changes}{\par\color{violet}\par\addvspace{\baselineskip}}{\par\addvspace{\baselineskip}}


%opening
\title{{\small Response Letter to the Editor concerning the Submission entitled:} \\ On the Prevalence of Condorcet's Paradox }
\subtitle{Ms. Ref. No.: EJ MS20240844}
\author{Salvatore Barbaro and Anna-Sophie Kurella}
\date{\today}
\begin{document}
\maketitle

\noindent Dear Amanda,

First of all, we would like to express our gratitude once again for the invitation to resubmit. You will notice that we have thoroughly revised the manuscript following a detailed consideration of the comments. In this letter, we aim to respond to your decision letter. In our separate \textit{Letter to the Reviewers}, we address the specific comments made by the referees.

Before addressing your questions, please allow us to briefly outline the main revisions compared to the initial version:
\begin{enumerate}
	\item The introduction now systematically presents the state of research and the research gap, linking the rather technical topic to recommendations for electoral reform.
	\item Data and empirical strategies are described more clearly, and the questions and critiques raised in your letter and the reviewers' comments are addressed.

	Following your recommendation to address questions of statistical inference, we applied bootstrap methods and a random-noise approach.
	
	While the initial version analysed all available party and candidate ratings, we now focus exclusively on real-world elections and use party ratings for parliamentary elections and candidate ratings for presidential elections. We analyse both types of data only if parliamentary and presidential elections occurred simultaneously. Additionally, we now utilise an updated version of the dataset that was recently published. 
	
	In the initial version, we interpreted ties as an indicator of cyclical majorities. Comments from the second reviewer regarding related literature led us to reconsider this approach and address it explicitly, which we have now done.
	\item Alongside our main finding (that the Condorcet Paradox has virtually no empirical relevance), we present a range of new insights regarding who the Condorcet winners are, how often they make it into government, and how various electoral systems influence these outcomes.
\end{enumerate}

\subsection*{1. How Novel is the Result?}
In the thoroughly revised introduction, we refer to existing research on the question at hand and explain why it has provided only limited insights into the prevalence of the paradox in large elections. In particular, existing approaches are mainly based on artificial data or non-political elections, which is based on systematically different preference data as democratic elections. We focus on real-world national elections, for which empirical evidence is only available for a handful of West European countries and the US.

We refer to the recent survey of empirical studies \citep{vanDeemen2013} and cite its conclusion that our knowledge of the relevance of the Condorcet Paradox remains unsettled. Furthermore, we highlight the research gap identified by Amartya \cite{Sen2017}, which calls for a cross-national empirical analysis. We believe that the marginal contribution of our paper is now outlined very clear.

You raised the question of whether our main finding is sufficiently surprising, given the prior conventional wisdom that the Condorcet Paradox occurs infrequently rather than commonly. We would like to address this concern as follows:
\begin{enumerate}
\item Empirical studies have repeatedly reported instances of cyclical majorities. While not described as a very frequent phenomenon (in contrast to the simulation-based research branch), it remained unclear how often these cases occur. Our central finding is novel because in national elections, cyclical majorities do not occur sometimes, occasionally, or 'not as frequently as expected,' but practically never.
%
\item The expectation of infrequent cases was based on empirical studies with several weaknesses, which we detail precisely in the paper's Introduction: the findings are limited to single countries and different types of elections (sub-national elections, primaries, referenda). Comparative empirical evidence on large-scale national elections is too scarce to draw conclusions on the prevalence of the Condorcet Paradox in national elections, especially beyond the boundaries of Western Europe and the US. Our contribution meets the standards set in the literature for a comprehensive study, encompassing multiple countries and various time periods.
%
\item Our analysis enables a range of observations that, to the best of our knowledge, are novel: Who are the Condorcet winners, and how effective are non-Condorcet voting methods at bringing Condorcet winners into political and administrative positions of responsibility? We also provide, as far as we know for the first time, insights into which party families are most frequently Condorcet winners and which parties are disproportionately affected by failing to join the government as Condorcet winners. In this context, we report differences between the various electoral \emph{systems} (FPTP, PR, Mixed), which, to the best of our knowledge, have not been analysed before. Furthermore, the current version also includes an analysis of the Condorcet efficiency of the Borda count, which, to our knowledge, has not been conducted in this form before. Finally, we also searched for cyclical majorities below the level of the Condorcet winners\footnote{
 As an example: there are cyclical majorities involving candidates $B$, $C$, and $D$; however, all voters prefer candidate $A$ the most.
} . In doing so, we analysed over 8,000 triplets and found that cyclical majorities occur only in absolute exceptional cases. Practically always, the social orderings resulting from simple majority rule are transitive (see lines \texttt{148$–$153}).

We believe that, overall, we offer a range of new insights, especially considering that we have presented these findings in a highly condensed format within a short paper.
\end{enumerate}

\subsection*{2. What Drives the Result?}
We have followed the recommendations of the second reviewer by thoroughly addressing the question of which parties are Condorcet winners and how often they achieve governmental responsibility under different electoral systems. However, as you rightly point out, this analysis does not answer the question of \textit{why} there are virtually no Condorcet Paradoxes. For such an analysis, we simply lack the necessary variance. Following your suggestion, we point at the role of single-peaked preferences, provide an additional analysis and encourage future researchers to address this question. You can find our discussion of this issue in lines (\texttt{180-196}), and results of a regression analysis in Appendix \textit{D}.

\subsection*{Empirical Strategy}

\begin{enumerate}[label=(\alph*)] 
	\item
\begin{quote}
$\ldots$ the paper provides too little information about the polling data used. $\ldots$ How many respondents are there?
\end{quote}	
Indeed, we provided too little information regarding the data and empirical strategy in the initial version. We are confident that the revised version comprehensively addresses this weakness. We refer to the new section \textit{"Data and Methods"} (Section 2), particularly paragraphs 1 to 4 and footnote 1. We have made an effort to highlight where some of the methods we applied have already been utilised in the existing literature.
%%%
\item 
\begin{quote}
	Do you use all available elections? Or have you selected 221? $\ldots$ If you have, what determined the selection?
\end{quote}
First, we would like to emphasise that, upon re-reading our initial draft, this question is absolutely justified. We regret that we were not sufficiently precise on this point either. We excluded only those cases for which data were available for fewer than three parties or candidates. For these cases, it does not make sense to examine cyclical majorities. Otherwise, the dataset is fully utilised, and no further selection takes place. 

The cases with only two ratings for parties or candidates amount to six in total. However, four of these are elections from the United States. We are aware that there is particular interest among the readership in analyses concerning the US. Unfortunately, the dataset does not allow for more in-depth US-related analysis. While we find this regrettable, we consider it justifiable. After all, the core of this work does not lie in individual case studies but rather in providing a comprehensive perspective on preferences and their aggregation.

Additionally, we had to remove the presidential elections in Kyrgyzstan 2005, as well as the Russian presidential elections in 2000 and 2004, due to a lack of like-dislike ratings of the presidential candidates Kurmanbek Bakiyev and Vladimir Putin, who ran as independents. We report these limitations in footnote 4.
\item \begin{quote}
	The paper assumes agents report truthfully. Why is this a good assumption in the context of this setting? $\ldots$ might the institution / might culture influence whether subjects report truthfully?
\end{quote}

Thank you for this suggestion, which we are happy to address here and in the text. The 'thermometer-style' (like/dislike) data we used have been employed in numerous studies as a strategy-proofed benchmark. They are compared with voting intentions or observed voting behaviour (in post-election surveys) to gain insights into strategic voting. To our knowledge, this approach was first introduced by \citet{Abramson2009}, and to this day \citep{Eggers2020, Eggers2022, Nunez2024, Eggers2024}, like/dislike data are considered strategy-proofed values. We have articulated this argument in lines \texttt{66$–$69}.

We reviewed the literature on 'thermometer-style' data and found no source that examines whether institutions or cultural motives influence truthful reporting. We believe that respondents did not face incentives to report untruthfully to party or candidate ratings in the CSES, even in authoritarian settings. Generally, the CSES data is regarded as of high-quality, adhering to scientific standards, and is used widely within the political science literature. The national surveys are usually embedded in a larger election study coordinated by non-for profit research institutes and collected by professional survey firms. We are convinced that this is the best available data source for comparative research on voting behaviour and political preferences.

Of course, the patterns of national preference profiles reflect to some extent the institutional design of a country's electoral system. The number of viable parties/candidates will be lower in majoritarian systems than in proportional systems. Respondents will likely express more indifferences between parties that are coalition partners, a pattern which we will rather not observe in presidential systems or Westminster-type systems. However, we do not think of these institutional influences as a disturbance parameter to the preference profiles, but are instead interested in the occurrence of Condorcet Paradoxes given these real-world empirical preference distributions over national political parties and candidates.


\item \begin{quote}
The paper appears to assume that preferences are measured accurately, i.e., there is no noise in reports or measurement 	error. This is far from a typical assumption. Why do we think it is a good assumption in this context?
\end{quote}
Thank you for this suggestion. We have expanded the methodological approach concerning statistical inference in two ways. Firstly, as already described, we used a bootstrap procedure to determine the probabilities of the paradox occurring. This addresses random variations in the sample.

Regarding the approach: for each sample (each election), we generated 10,000 bootstrap replications and examined how often a paradox emerged within these replications. We followed a recommendation by the second reviewer, who referred to \cite{Darmann2019}, where a similar procedure was applied. The results demonstrate that our central finding is highly robust. To avoid repetition, we kindly refer to the relevant passages in Section 3 of the paper.

Secondly, we added random-noise values to the ratings of parties or candidates. Again, to avoid duplication, we refer to the last two paragraphs of Section 2 of the text. The main effect of the random-noise procedure is that we effectively (except for extremely rare cases) eliminate all instances of indifference. For each election, we generated 10,000 new replications, resulting in 2.53 million checks for the occurrence of cyclical majorities. Similar to the bootstrap analysis results, we conclude that cyclical majorities occur very rarely.

\item \begin{quote}
As Referee 3 points out, it is not terribly helpful to point to what R package you use. Much more helpful would be a discussion of what is actually done.
\end{quote}
Thank you for this valid comment. The paper now presents the methodological approach in a precise manner. We refer to the discussion starting from line \texttt{92} (up to line \texttt{104}).
\end{enumerate}

\subsection*{Modest Comments}
\begin{quotation}
 As the referees note, the paper needs to do a better job of writing for a general interest audience.
\end{quotation}

We have taken this comment very seriously and comprehensively revised the paper in light of it. The relevance and significance of the research question should now be clear from the introduction, as well as the novelty of our findings. We have eliminated overly field-specific terminology and moved it to footnotes where necessary. Additionally, we incorporated the suggestion of the second reviewer to highlight the relevance for policy advising already in the introduction.

\begin{quotation}
I did not find the discussion of the IAC particularly helpful. You use it to contrast \textit{actual}
preference profiles from the fraction of potential preference profiles. I am not sure that the
latter is what a researcher would jump to in 2024.
\end{quotation}

Thank you for this valid comment. As you can see, the discussion regarding the research area of simulations and probability calculations has now been significantly reduced. We note that this area has existed and continues to exist (around line \texttt{22}), but it has also been clear for several years that only a comprehensive empirical investigation can meet contemporary standards.

\hspace{0.8cm}

Esteemed Editor, 

We hope to have adequately addressed the three major issues. In a separate document, we provide a detailed response to each aspect raised by the three reviewers. We have taken the liberty of drafting a single letter addressing all three reviewers. Please let us know if you would prefer three separate letters instead. Otherwise, we would like to express our gratitude for the thoughtful and constructive feedback on our submission. It has been a pleasure revising our manuscript, and we hope you will continue to consider our work. 


Best regards

\hspace{.6cm}

Salvatore Barbaro and Anna-Sophie Kurella.

\newpage

\bibliographystyle{elsarticle-harv}
\bibliography{CondorcetCycle}
\end{document}
