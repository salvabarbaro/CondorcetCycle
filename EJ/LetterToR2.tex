\documentclass[a4paper, 12pt]{scrartcl}
% This version: 10 JAN, 11:21
\usepackage[utf8]{inputenc}
\usepackage{lmodern}
\usepackage{xcolor}
%\usepackage{setspace}
\usepackage{lineno}
\usepackage{amsmath}
\usepackage{amsfonts}
\usepackage{amssymb}
\usepackage{mathrsfs}
\usepackage{graphicx}
\usepackage[nice]{nicefrac}
\usepackage{natbib}
\usepackage{ntheorem}
\usepackage{authblk}
\usepackage[breaklinks]{hyperref}
\hypersetup{colorlinks, 
	linkcolor=blue, 
	citecolor=blue,
	urlcolor=blue}
\theoremstyle{break}
%\usepackage{orcidlink}
\newtheorem{Axiom}{Axiom}[section]
\newtheorem{lemma}{Lemma}[section]
\newtheorem{definition}{Definition}[section]
\newtheorem{theorem}{Theorem}[section]
\newtheorem{proof}{Proof}[lemma]
\newtheorem{corollary}{Corollary}[lemma]
\newtheorem{property}{Property}[section]
\newenvironment{changes}{\par\color{violet}\par\addvspace{\baselineskip}}{\par\addvspace{\baselineskip}}


%opening
\title{{\small Response memo for reviewer R2 concerning the submission entitled:} \\ On the Prevalence of Condorcet's Paradox }
\subtitle{Ms. Ref. No.: EJ MS20240844}
\author{Salvatore Barbaro and Anna-Sophie Kurella}
\date{\today}
\begin{document}
\maketitle

%\linenumbers

\noindent Esteemed Reviewer,

First and foremost, we would like to express our heartfelt gratitude for the very constructive and exceptionally profound evaluations. Your assessments were highly motivating. Your insightful comments have significantly contributed to enhancing the quality and rigour of our work. In response, we have thoroughly revised the manuscript.

For the sake of easier reference, I have taken the liberty of highlighting your comments in colour. This way, you can easily see how we have addressed the specific remarks. For the sake of improved readability, we have taken the liberty of summarising some comments briefly. This is by no means intended to imply a judgment, but rather to focus on the central essence of each remark.

Before diving into details, we would like to provide an overview of the significant changes compared to the initial version.

Before addressing your questions, please allow us to briefly outline the main revisions compared to the initial version:
\begin{enumerate}
\item The introduction now systematically presents the state of research and the research gap, linking the rather technical topic to recommendations for electoral reform.
\item Data and empirical strategies are described more clearly, and the questions and critiques raised your reports are addressed.
	
Following an advice of the editor to address questions of statistical inference, we applied bootstrap methods and a random-noise approach.
	
While the initial version analysed all available party and candidate ratings, we now focus exclusively on real-world elections and use party ratings for parliamentary elections and candidate ratings for presidential elections. We analyse both types of data only if parliamentary and presidential elections occurred simultaneously. Additionally, we now utilize an updated version of the dataset that was recently published. 
	
	
In the initial version, we interpreted ties as an indicator of cyclical majorities. Comments from the second reviewer regarding related literature led us to reconsider this approach and address it explicitly, which we have now done.
%
\item Alongside our main finding (that the Condorcet Paradox has virtually no empirical relevance), we present a range of new insights regarding who the Condorcet winners are, how often they make it into government, and how various electoral systems influence these outcomes.
\end{enumerate}

\section*{Response Memo}
\subsection*{Main Comments}
\begin{changes}
	The result is very interesting with great data but makes the reader wanting a thorough discussion. For example, the paper would benefit from a mechanism exploration for general interest journal like EJ. In some sense it’s a descriptive analysis of the Condorcet issue (which is an important first step), but the next step is to determine what factors correlate with it. There is some exploratory analysis along this line, but is based only on party. It would be interesting to also analyze countries / institutional differences. The authors should investigate what specific factors (e.g., electoral systems, political culture, party fragmentation) correlate with the presence or absence of Condorcet winners. A first start would be to look at other work; do the papers
	that this paper improves over do anything more than just counting winners/losers? I understand that finding variables to study mechanisms in this cross-country context may be very difficult.
\end{changes}

Thank your for raising the point of mechanism exploration, which we agree to be an important point. You are right in stating that our analysis—specifically regarding the frequency of cyclical preferences— remains descriptive and does not answer the question of \textit{why} there are virtually no Condorcet Paradoxes. For such an analysis, we simply lack the necessary variance. Since we also find this unsatisfactory, we discuss the role of single-peakedness of the preference profiles in leading to the non-occurrence of Condorcet Paradoxes in national elections. We conduct an additional analysis (presented in the Online appendix \textit{D }and described in lines \texttt{192$-$200}), to investigate to what extent individual party ratings are shaped by ideological left-right distances. The results reveal that left-right distance between respondent and party is a significant predictor of party ratings throughout the whole sample. The effect is general, and there is no significant variation across countries. Although we cannot prove that this exclusively explains the absence of Condorcet paradoxa, it suggests that preference profiles follow ideological orderings of parties, and that this leads to a sufficiently large number of single-peaked preferences within each national sample, that prevents the existence of a majority cycle. This may differentiate our data from other research using either artificial data or data on non-political elections, that more frequently detect majority cycles. Yet, since we find practically no instance of a Condorcet Paradox, we cannot provide any statistical test on this relation, and leave it for future research to investigate the underlying mechanism. We include a brief discussion of the potential mechanism in lines \texttt{182$-$191}.

\begin{changes}
	Related to the last point, how do the results change compared to the older papers that use worse data? Understanding the bias of those papers and how it shows up in the results would be illuminating (and boost the argument for contribution). I understand this may be difficult if the elections don’t overlap (like their discussion on the Van	Deemen and Vergunst), but just an overall result comparison may be useful.
\end{changes}
The answer to this point is related to the previous one. We focus on real-world national elections, which is based on systematically different preference data than non-political elections or simulated preference data. This may be one explanation for the rare occurrence of Condorcet Paradoxes, but we cannot prove this claim. Considering other research on political elections, we explicitly state where our sample overlaps, and where not. Unfortunately, our data do not cover the Danish 1994 election and the 2016 US election, for which existing research suggests majority cycles may have existed. For all other overlapping cases, we reach the same conclusion as existing research. Please refer to lines \texttt{215 $-$ 226}.

\begin{changes}
Why is this relevant beyond theoretical interest? Is there evidence policymakers take Condorcet paradox into account when devising voting rules? If so, you can quantify the extent of the over-correction. Overall, more discussion and some real-world policy relevance / recommendation would benefit the reader.
\end{changes}

Thank you for raising this important point. At the beginning of the manuscript and in the Conclusion, we connect our findings to the ongoing discussion on electoral reforms. In particular, we point to authors advocating for the Condorcet method \citep{Maskin2016, Maskin2017, Maskin2017a}. We argue that the Condorcet method has often been dismissed prematurely due to concerns about indeterminacy; however, our results demonstrate that this issue is not empirically significant. In particular, we highlight the findings in Table \textit{2} regarding the Condorcet efficiency of various electoral systems, which we believe should be carefully considered when drafting electoral reforms. We hope the practical relevance of this topic is evident.

\begin{changes}
If the number of Condorcet paradoxes in the data is ``surprisingly'' small (from a theory perspective), what explains the discrepancy? Are there nuances to the electoral systems that make them more complex than the idealized models?
\end{changes}

We trust that this point is addressed in our response to your first query regarding the underlying mechanism driving the results. To reiterate, we do not believe the key factor lies in the electoral systems, as there is no significant variation across them. Instead, the structure of preference profiles appears to be predominantly shaped by a single latent dimension. As demonstrated by the additional regression analysis in Online Appendix \textit{D}, this dimension is strongly associated with ideology. If a sufficiently large proportion of voters exhibit single-peaked preferences along this underlying dimension, a Condorcet winner will emerge \citep{Black1958}. This explanation seems the most plausible for the absence of Condorcet paradoxes, particularly in contrast to non-political elections.

\begin{changes}
For a general interest audience, more discussion on the Condorcet paradox (winner/loser) is needed. The concise definition in the intro doesn’t feel sufficient.
\end{changes}

We agree and revised the introduction. The Introduction now presents a standard definition of the Condorcet Paradox, trying to avoid technical jargon as remarked by Reviewer \# $3$.

\subsection*{Minor Comments}
\begin{changes}
More details on data. For example, why just 1996/2004 for the US? The reform party only did well in 1992/1996. What about Nader in 2000 with the greens? Is this scattered years list simply based on what CSES has (my guess for why 1992 is missing) or are there additional data processing steps that removed some years?
\end{changes}
Thank you for highlighting the weaknesses in our data description. We agree that the original manuscript did not sufficiently detail the utilised data. To address this, we have included a new section, "2. Data and Methods" which is dedicated to describing our data and research strategy. 

To clarify, we used all available data for elections involving more than two parties or candidates, excluding only those elections for which essential data was unavailable. Unfortunately, this includes most US presidential elections, as the CSES often only provides ratings for Democratic and Republican candidates, even when more than two candidates were on the ballot. Similarly, we excluded the presidential elections in Kyrgyzstan and Russia due to the absence of ratings for the most viable candidates, Kurmanbek Bakiyev and Vladimir Putin, respectively.

We explicitly address these data limitations in the main text and provide details on the number of respondents, countries, and elections included in our final dataset (see lines \texttt{86$-$93}). Additionally, we describe the bootstrap and random noise approaches employed to account for uncertainty in our data (lines \texttt{114$-$142}). We hope that this newly added section offers all the necessary details to follow our analysis. Furthermore, we will make all replication materials publicly available.

\begin{changes}
Other recent work that may be relevant for authors to compare to: \cite{Desai2025, Darmann2019}.
\end{changes}

Thanks for suggesting additional literature, that we indeed have not been aware of. Both papers were highly inspiring for our revision, especially regarding their approaches to statistical inference. We employ a similar bootstrap approach as \citeauthor{Darmann2019} and cite both references accordingly in the main text.  We were pleased to see that \cite{Desai2025} also utilized the like/dislike data from the CSES. Their note on the necessity of splitting the data for Belgium was particularly helpful and saved us from making a mistake. Thank you for your valuable hints.

\begin{changes}
Alternative measure / aggregation rule to see how things change?
\end{changes}

Thank you for this interesting suggestion. We have not only calculated the Condorcet efficiency for various voting systems (see Table \textit{2}, as described above) but also analysed how it compares under the plurality rule and Borda rule. The values for the plurality rule are presented in Table \textit{2}, while the Condorcet efficiency of  Borda's rule was determined by re-evaluating all data according to Borda's scoring rule. There is extensive literature on the debate between 'Condorcet vs. Borda'. Our analysis highlights that the outcomes of these two approaches often align, and we explain the circumstances under which they diverge. For further details, please refer to lines \texttt{285$–$300}. 

\begin{changes}
The mention of non-reporting of null results is interesting but is the claim that some authors sought evidence, presented, but did not end up publishing results a guess or based on knowledge? If the latter, the presentations could be cited.
\end{changes}

We have removed this part, because, indeed, the claim is a guess.

\begin{changes}
Typo in Figure 1 for Conservative.
\end{changes}
Thank you for pointing on this. We have amended the caption. 

\section*{Concluding Remarks}
We would like to once again express our sincere gratitude for your exceptionally balanced and constructive remarks. We truly hope that the new version not only incorporates your valuable suggestions for improvement but also exhibits an overall increase in quality. We are eagerly looking forward to your feedback.

\bibliographystyle{elsarticle-harv}
\bibliography{CondorcetCycle}
\end{document}
