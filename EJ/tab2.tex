\begin{table}[ht]
\caption{Condorcet efficiency by type of election (parliamentary vs. presidential and by electoral systems.}
\centering
\begin{tabular}{l|rrr|r} \toprule 
   & \multicolumn{3}{c|}{\textbf{Parliamentary}} & \textbf{Presidential}\\
 Condorcet Winner& Plurality  & Proportional & Mixed &  \\
                &   $N=26$   & $N=135$   & $N=51$    & $N=41$ \\ \midrule 
% row 1: Largest elec. party / candidate                
largest elect. & 68\%     & 61\%   &   82\% &  79\%  \\
party / candidate & \emph{[51-81]} & \emph{[54-68]} & \emph{[72-90]} & \emph{[67-88]} \\ \midrule 
% row 2: PM
prime minister/ &   88\% &  66\% &   73\% &   78\% \\
president & \emph{[73-96]} & \emph{[58-72]} & \emph{[61-82]} & \emph{[64-87]} \\ \midrule
% row 3: cabinet participation
part of &   92\% &  89\% &   98\% &    \\
government & \emph{[78-98]} & \emph{[84-93]} & \emph{[91-100]} & \\ \midrule
%  row 4: CL
Condorcet Loser & & & & \\
part of & 0\%& 13\%& 6\% & 2\%\\
government / president &  & \emph{[11-23]} & \emph{[3-19]} & \emph{[0-11]} \\
\bottomrule 
\end{tabular}
\label{tb.efficiency}
\end{table}