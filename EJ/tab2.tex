\begin{table}[ht]
\caption{Condorcet efficiency by type of election (parliamentary vs. presidential and by electoral systems.}
\centering
\begin{tabular}{l|rrr|r} \toprule 
 %  & \multicolumn{4}{c}{Type of election and electoral system}\\
   & \multicolumn{3}{c|}{\textbf{Parliamentary}} & \textbf{Presidential}\\
 Condorcet Winner& Plurality  & Proportional & Mixed &  \\
                &   N=30   & N=135   & N=51    & N=46 \\ \midrule 
largest elect.& 71\%     & 62\%   &   81\% &  82\%  \\
party / candidate & \emph{[56-83]} & \emph{[54-68]} & \emph{[70-88]} & \emph{[71-89]} \\ \midrule 
% party/candidate  &  (20/28) &  (82/133) &  (39/48) &  (37/45) \\ \midrule 
   prime minister/ &   89\% &  66\% &   73\% &   82\% \\
president & \emph{[75-96]} & \emph{[58-73]} & \emph{[61-82]} & \emph{[71-90]} \\ \midrule
%  president         &  (25/28) & (76/115) &  (35/48) &  (37/45) \\  \midrule
part of &   97\% &  88\% &   98\% &    \\
government & \emph{[85-100]} & \emph{[82-92]} & \emph{[91-100]} & \\ \midrule
%  government &  (29/30) & (119/135) &  (50/51) &   \\ \midrule
Condorcet Loser & & & & \\
part of & 0\%& 16\%& 8\% & 4\%\\
government / president &  & \emph{[11-22]} & \emph{[2-19]} & \emph{[0-12]} \\
%government/president & (0/20) & (18/112) & (3/37) & (2/47) \\
\bottomrule 
\end{tabular}
\label{tb.efficiency}
\end{table}