\documentclass[a4paper, 12pt]{scrartcl}
%
\usepackage[utf8]{inputenc}
\usepackage{lmodern}
\usepackage{xcolor}
%\usepackage{setspace}
\usepackage{lineno}
\usepackage{amsmath}
\usepackage{amsfonts}
\usepackage{amssymb}
\usepackage{mathrsfs}
\usepackage{graphicx}
\usepackage[nice]{nicefrac}
%\usepackage{xfrac}
%\usepackage{marvosym}
%\usepackage{circledsteps}
\usepackage{natbib}
\usepackage[titletoc]{appendix}
\usepackage{ntheorem}
\usepackage{authblk}
\usepackage[breaklinks]{hyperref}
\hypersetup{colorlinks, 
	linkcolor=blue, 
	citecolor=blue,
	urlcolor=blue}
\theoremstyle{break}
%\usepackage{orcidlink}
\newtheorem{Axiom}{Axiom}[section]
\newtheorem{lemma}{Lemma}[section]
\newtheorem{definition}{Definition}[section]
\newtheorem{theorem}{Theorem}[section]
\newtheorem{proof}{Proof}[lemma]
\newtheorem{corollary}{Corollary}[lemma]
\newtheorem{property}{Property}[section]
\newenvironment{changes}{\par\color{violet}\par\addvspace{\baselineskip}}{\par\addvspace{\baselineskip}}


%opening
\title{{\small Response memo concerning the submission entitled:} \\ On the Prevalence of Condorcet's Paradox }
\subtitle{Ms. Ref. No.: EJ MS20240844}
\author{Salvatore Barbaro and Anna-Sophie Kurella}
\date{\today}
\begin{document}
\maketitle

\linenumbers

\noindent Esteemed Reviewers,


First and foremost, we would like to express our heartfelt gratitude for the very constructive and exceptionally profound evaluations. Your assessments were highly motivating. Your insightful comments have significantly contributed to enhancing the quality and rigour of our work. In response, we have thoroughly revised the manuscript.

For the sake of easier reference, I have taken the liberty of highlighting your comments in colour. This way, you can easily see how we have addressed the specific remarks. For the sake of improved readability, we have taken the liberty of summarising some comments briefly. This is by no means intended to imply a judgment, but rather to focus on the central essence of each remark.

Before diving into details, we would like to provide an overview of the significant changes compared to the initial version.

Before addressing your questions, please allow us to briefly outline the main revisions compared to the initial version:
\begin{enumerate}
\item The introduction now systematically presents the state of research and the research gap, linking the rather technical topic to recommendations for electoral reform.
\item Data and empirical strategies are described more clearly, and the questions and critiques raised your reports are addressed.
	
Following an advice of the editor to address questions of statistical inference, we applied bootstrap methods and a random-noise approach.
	
While the initial version analysed all available party and candidate ratings, we now focus exclusively on real-world elections and use party ratings for parliamentary elections and candidate ratings for presidential elections. We analyse both types of data only if parliamentary and presidential elections occurred simultaneously. Additionally, we now utilize an updated version of the dataset that was recently published. 
	
	
In the initial version, we interpreted ties as an indicator of cyclical majorities. Comments from the second reviewer regarding related literature led us to reconsider this approach and address it explicitly, which we have now done.
%
\item Alongside our main finding (that the Condorcet Paradox has virtually no empirical relevance), we present a range of new insights regarding who the Condorcet winners are, how often they make it into government, and how various electoral systems influence these outcomes.
\end{enumerate}

%\begin{enumerate}
% \item We reformulate the introduction, avoiding technical jargon, to make the research note appealing to a wider audience. The introduction concisely reports the current state of research, highlights the expressed demand for a comprehensive empirical study (Sen 2017), whereby making clear the novelty of our research.
%\item We carefully revised our data and clearly separate parliamentary and presidential elections in our analyses. We utilised an updated and enhanced version of the data set, which could not considered for the initial version. The revised version focuses on 212 parliamentary elections and 41 presidential elections. 
%\item Based on your remarks, we add a systematic analysis of the performance of Condorcet winners and losers across election type and electoral formulae, differentiating for electoral victory and winning different types of office. Aside from the main insight that the Condorcet paradox has virtually no empirical relevance, this part is also a pioneering assessment on how the Condorcet-winner parties achieve political influence. 
%\item We also assess Condorcet losers and provide evidence for the so-called Borda paradox (where the Condorcet loser emerges as election winner). Furthermore, we relate our results more accurately to existing research and policy relevance.
%\end{enumerate}

\newpage
\section*{R1} 
\begin{changes}
1. Should we not worry that the preferences reported in polls are affected by the electoral system? I do not worry here about misrepresenting true preferences, but my preference over parties may be different in a system with majority voting where the winner then governs unchallenged than in a parliamentary system where I expect even the winner party to have to form a coalition? It would be good to have some discussion of this.
\end{changes}

We agree that the patterns of national preference profiles reflect to some extend the institutional design of a country's electoral system. The number of viable parties/candidates will be lower in majoritarian systems than in proportional systems, and we know from the literature that the number of candidates increases the likelihood of a Condorcet Paradox. However, respondents will likely express more indifferences between parties that are ideologically close and/or coalition partners in large party sytems, a pattern which we will rather not observe in presidential systems or Westminster-type systems. Indifferences reduce the likelihood of a Condorcet Paradox. Therefore it is hard to assess how the institutional impact on preference profiles will influence the occurrence of Condorcet Paradoxes. 

We thus acknowledge that the electoral system affects party preferences, and that this might affect the likelihood of majority cycles. However, we do not see this as a bias to our results. We do not regard institutions as a disturbance parameters to  preference profiles, but  instead we are interested in the occurrence of Condorcet Paradoxes given real-world institutions and the resulting empirical preference distributions over national political parties and candidates. To be clear, our aim is not to make claims about how voters would vote under the Condorcet method as an alternative to their national electoral system, but we want to investigate whether their true, empirical preferences over the national parties or candidates lead to a majority cycle. And indeed our results show that Condorcet Paradoxes do practically not occur under any electoral institution in our sample. For the sake of brevity, we refrain from discussing this point in the research note, and instead focus on the Condorcet efficiency of electoral institutions, where do detect significant variation.

\begin{changes}
2. I like that the authors attempt to characterize who the Condorcet winners typically are, and in particular, how much support the Condorcet losers typically get. I thought, though that this part of the paper could be a bit more expanded. For example, the last sentence of the conclusion says that "Condorcet loser parties often benefit from the actual voting methods", but the paper only gives one example of such a thing happening. Some quantification of this statement in the paper would be helpful. Similarly, there is an example of D66, a party that was a Condorcet winner but received very few votes. How frequent are such examples?
\end{changes}

\begin{changes}
3. I enjoyed learning that in 85\% of the election sight a Condorcet winner that party ended up in the government. I have two questions, though. First, what does "ending up in the government" mean? How many times did the Condorcet winner become the coalition leader, that is, won the highest number of seats? And if it is close to 85\%, I would like to see some discussion how likely this is far from 100\%. I understand that one cannot put a standard confidence interval on that, but perhaps one can have a discussion based on the typical margin of error that the polls are expected to have and how frequently the election results actually defy the polls. 
\end{changes}

Due to the substantive connection between the two points and to avoid repetition, we have taken the liberty of addressing them together.

Thank you for raising this question, which we aim to address satisfactorily in (and around) Table \textit{2} of the revised manuscript. The table provides a detailed breakdown of the frequencies at which Condorcet winners emerge as the largest electoral party, assume the position of prime minister or president, and hold any government post. Additionally, we present % bootstrap
confidence intervals
%
from Agresti-Coull binomial tests
%
to account for the statistical uncertainty. We hope that the insights offered by Table \textit{2} will be of considerable interest to a wider audience, and -- again -- we are grateful for your valuable suggestion.

Concretely, we expand the discussion of the electoral performance of selected Condorcet winners and losers in lines \texttt{217$-$233}. The data in Table \textit{2} show that government participation of Condorcet losers is indeed quite rare, and most likely occurs in systems of proportional representation. There is one instance where the Condorcet loser become president. We discuss these results lines \texttt{266$-$280} in the main text.


\newpage
\section*{R2}
\subsection*{Main Comments}
\begin{changes}
	The result is very interesting with great data but makes the reader wanting a thorough discussion. For example, the paper would benefit from a mechanism exploration for general interest journal like EJ. In some sense it’s a descriptive analysis of the Condorcet issue (which is an important first step), but the next step is to determine what factors correlate with it. There is some exploratory analysis along this line, but is based only on party. It would be interesting to also analyze countries / institutional differences. The authors should investigate what specific factors (e.g., electoral systems, political culture, party fragmentation) correlate with the presence or absence of Condorcet winners. A first start would be to look at other work; do the papers
	that this paper improves over do anything more than just counting winners/losers? I understand that finding variables to study mechanisms in this cross-country context may be very difficult.
\end{changes}

Thank your for raising the point of mechanism exploration, which we agree to be an important point. You are right in stating that our analysis—specifically regarding the frequency of cyclical preferences— remains descriptive and does not answer the question of \textit{why} there are virtually no Condorcet Paradoxes. For such an analysis, we simply lack the necessary variance. Since we also find this unsatisfactory, we discuss the role of single-peakedness of the preference profiles in leading to the non-occurrence of Condorcet Paradoxes in national elections. We conduct an additional analysis (presented in the Online appendix \textit{D }and described in lines \texttt{192$-$200}), to investigate to what extent individual party ratings are shaped by ideological left-right distances. The results reveal that left-right distance between respondent and party is a significant predictor of party ratings throughout the whole sample. The effect is general, and there is no significant variation across countries. Although we cannot prove that this exclusively explains the absence of Condorcet paradoxa, it suggests that preference profiles follow ideological orderings of parties, and that this leads to a sufficiently large number of single-peaked preferences within each national sample, that prevents the existence of a majority cycle. This may differentiate our data from other research using either artificial data or data on non-political elections, that more frequently detect majority cycles. Yet, since we find practically no instance of a Condorcet Paradox, we cannot provide any statistical test on this relation, and leave it for future research to investigate the underlying mechanism. We include a brief discussion of the potential mechanism in lines \texttt{180$-$196}.

%Instead, we focus on Condorcet efficiency across electoral systems and reveal significant variation between parliamentary and presidential elections, as well as between different electoral formulae. For the sake of brevity given the format of a research note, we keep these analyses at a descriptive level and hope that the variation we detect stimulates further research on institutional drivers of Condorcet efficiency.  

\begin{changes}
	Related to the last point, how do the results change compared to the older papers that use worse data? Understanding the bias of those papers and how it shows up in the results would be illuminating (and boost the argument for contribution). I understand this may be difficult if the elections don’t overlap (like their discussion on the Van	Deemen and Vergunst), but just an overall result comparison may be useful.
\end{changes}
The answer to this point is related to the previous one. We focus on real-world national elections, which is based on systematically different preference data than non-political elections or simulated preference data. This may be one explanation for the rare occurrence of Condorcet Paradoxes, but we cannot prove this claim. Considering other research on political elections, we explicitly state where our sample overlaps, and where not. Unfortunately, our data do not cover the Danish 1994 election and the 2016 US election, for which existing research suggests majority cycles may have existed. For all other overlapping cases, we reach the same conclusion as existing research that no Paradox occurred.



\begin{changes}
	Why is this relevant beyond theoretical interest? Is there evidence policymakers take Condorcet paradox into account when devising voting rules? If so, you can quantify the extent of the over-correction. Overall, more discussion and some real-world policy relevance / recommendation would benefit the reader.
\end{changes}

Thank you for raising this important point. At the beginning of the manuscript and in the Conclusion, we connect our findings to the ongoing discussion on electoral reforms. We argue that the Condorcet method has often been dismissed prematurely due to concerns about indeterminacy; however, our results demonstrate that this issue is not empirically significant. In particular, we highlight the findings in Table 2 regarding the Condorcet efficiency of various electoral systems, which we believe should be carefully considered when drafting electoral reforms. We hope the practical relevance of this topic is evident.

\begin{changes}
	If the number of Condorcet paradoxes in the data is ``surprisingly'' small (from a theory perspective), what explains the discrepancy? Are there nuances to the electoral systems that make them more complex than the idealized models?
\end{changes}

We trust that this point is addressed in our response to your first query regarding the underlying mechanism driving the results. To reiterate, we do not believe the key factor lies in the electoral systems, as there is no significant variation across them. Instead, the structure of preference profiles appears to be predominantly shaped by a single latent dimension. As demonstrated by the additional regression analysis in Online Appendix D, this dimension is strongly associated with ideology. If a sufficiently large proportion of voters exhibit single-peaked preferences along this underlying dimension, a Condorcet winner will emerge. This explanation seems the most plausible for the absence of Condorcet paradoxes, particularly in contrast to non-political elections.

\begin{changes}
	For a general interest audience, more discussion on the Condorcet paradox (winner/loser) is needed. The concise definition in the intro doesn’t feel sufficient.
\end{changes}

We agree and revised the introduction. The introduction now presents a standard definition of the Condorcet Paradox, trying to avoid technical jargon as remarked by Reviewer No. 3.

\subsection{Minor Comments}
\begin{changes}
	More details on data. For example, why just 1996/2004 for the US? The reform party only did well in 1992/1996. What about Nader in 2000 with the greens? Is this scattered years list simply based on what CSES has (my guess for why 1992 is missing) or are there additional data processing steps that removed some years?
\end{changes}
Thank you for highlighting the weaknesses in our data description. We agree that the original manuscript did not sufficiently detail the data utilised. To address this, we have included a new section, "2. Data and Methods" which is dedicated to describing our data and research strategy. To clarify, we used all available data for elections involving more than two parties or candidates, excluding only those elections for which essential data was unavailable. Unfortunately, this includes most US presidential elections, as the CSES often only provides ratings for Democratic and Republican candidates, even when more than two candidates were on the ballot. Similarly, we excluded the presidential elections in Kyrgyzstan and Russia due to the absence of ratings for the most viable candidates, Kurmanbek Bakiyev and Vladimir Putin, respectively.

We explicitly address these data limitations in the main text and provide details on the number of respondents, countries, and elections included in our final dataset. Additionally, we describe the bootstrap and random noise approaches employed to account for uncertainty in our data. We hope that this newly added section offers all the necessary details to follow our analysis. Furthermore, we will make all replication materials publicly available.

\begin{changes}
	Other recent work that may be relevant for authors to compare to: \cite{Desai2025, Darmann2019}.
\end{changes}

Thanks for suggesting additional literature, that we indeed have not been aware of. Both papers were highly inspiring for our revision, especially regarding their approaches to statistical inference. We employ a similar Bootstrap approach as \citeauthor{Darmann2019} and cite both references accordingly in the main text. 


\begin{changes}
Alternative measure / aggregation rule to see how things change?
\end{changes}

Thank you for this interesting suggestion. We have not only calculated the Condorcet efficiency for various voting systems (see Table 2, as described above) but also analysed how it compares under the plurality rule and Borda rule. The values for the plurality rule are presented in Table 2, while the Condorcet efficiency of  Borda's rule was determined by re-evaluating all data according to Borda's scoring rule. There is extensive literature on the debate between 'Condorcet vs. Borda'. Our analysis highlights that the outcomes of these two approaches often align, and we explain the circumstances under which they diverge. For further details, please refer to lines 285–300. 

\begin{changes}
The mention of non-reporting of null results is interesting but is the claim that some authors sought evidence, presented, but did not end up publishing results a guess or based on knowledge? If the latter, the presentations could be cited.
\end{changes}

We have removed this part, because, indeed, the claim is a guess.

\begin{changes}
	Typo in Figure 1 for Conservative.
\end{changes}
Thank you for pointing on this. We have amended the caption. 

\newpage
\section{R3}
\begin{changes}
My main concern is that the current manuscript looks more like a short technical note addressed to an audience of readers who already know the topic really well, rather than a paper addressed to a broad audience of economists. The results seem interesting and potentially important, but I believe the paper would have to be written completely differently to do them justice.
\end{changes}

First of all, thank you for this critical yet very valid remark. Following your advice, we have thoroughly revised the entire paper, guided by the aim of creating an accessible, engaging piece filled with new findings that appeals to a broad spectrum of economists.


\begin{changes}
The authors could start by explaining in “simple” words what a Condorcet winner is and provide concrete examples to illustrate the Condorcet paradox. In its current form, the introduction immediately starts with some jargon that many readers who do not work in social choice theory will not know. Although most economists have heard the term Condorcet paradox before, I believe it is worth reminding everyone what that is before testing for its existence.
\end{changes}

Thanks for pointing at this weakness. We revised the introduction by removing much of the social-choice jargon, providing a standard definition of the Condorcet Paradox and mentioning its significance to economic theory (stability of core economy). 


\begin{changes}
There is very little information about the empirical strategy. I think it would be important to provide more information on the polling data that is used (e.g., who conducts these surveys, how many respondents there are, what the sampling strategy is, etc.). It would also be interesting to know how the parties are classified between the different categories shown in Figure 1. Is it performed by the authors? Based on which sources?
\end{changes}

Thank you for highlighting the weaknesses in our data description. We agree that the original manuscript did not sufficiently detail the data utilised. To address this, we have included a new section, "2. Data and Methods", in the revision, which is dedicated to describing our data and research strategy. It starts with giving a short description on the CSES data. 
% Anna: das würde ich gerne weglassen:
%Since it is a joint program of multiple national elections study teams, the sampling procedures vary between countries and over time. Given the brevity of the research note, we cannot provide details on each national subset of the data. However, all details are documented at their website, to which we direct the interested reader in footnote 1. 

We further provide details on the number of respondents, countries, and elections included in our final dataset, as well as on the way these survey are collected. Additionally, we describe the bootstrap and random noise approaches employed to account for uncertainty in our data. We hope that this newly added section offers all the necessary details to follow our analysis. Furthermore, we will make all replication materials publicly available.

\begin{changes}
Similarly, it would be valuable to explain in more detail which calculations are performed to determine whether there is a Condorcet winner or not. The authors mention an R package but I think it would be important to explain what this package does exactly.
\end{changes}
We agree, and outline the method to determine Condorcet winners and losers in more detail in lines 101-113.

\begin{changes}
	On a more substantive note, isn’t there a risk that the party ratings given by respondents are endogenously influenced by the electoral system in which these respondents vote? It could be nice to defend a bit more the assumption that these ratings can be treated as sincere voter preferences.
\end{changes}
Since this point has also been raised by Reviewer No. 1 we partly copy our reply from above: 

We agree that the patterns of national preference profiles reflect to some extend the institutional design of a country's electoral system. The number of viable parties/candidates will be lower in majoritarian systems than in proportional systems, and we know from the literature that the number of candidates increases the likelihood of a Condorcet Paradox. However, respondents will likely express more indifferences between parties that are ideologically close and/or coalition partners in large party systems, a pattern which we will rather not observe in presidential systems or Westminster-type systems. Indifferences reduce the likelihood of a Condorcet Paradox. Therefore it is hard to assess how the institutional impact on preference profiles will influence the occurrence of Condorcet Paradoxes. 

We acknowledge that the electoral system influences party preferences and may impact the likelihood of majority cycles. However, we do not view this as a bias in our results. Rather than considering institutions as mere disturbance factors to preference profiles, we focus on the occurrence of Condorcet paradoxes within the context of real-world institutions and the resulting empirical preference distributions across national political parties and candidates. To clarify, our aim is not to speculate on how voters might behave under the Condorcet method as an alternative to their national electoral system. Instead, we seek to examine whether their actual, empirically observed preferences for national parties or candidates result in a majority cycle. And, indeed, our results show that Condorcet Paradoxes do practically not occur under any electoral institution. For the sake of brevity, we refrain from discussing this point in the research note, and instead focus on the Condorcet efficiency of electoral institutions, where do detect significant variation.

In response to your concern that the like-dislike ratings may not accurately reflect sincere preferences, we address this by referencing existing studies that utilize like-dislike ratings to investigate strategic voting. This literature commonly assumes that these ratings serve as strategy-proof reference values, against which actual vote intentions can be compared. Given that this approach is widely accepted in the extensive literature on strategic voting, we anticipate little disagreement in relying on the assumption of truthful reporting for party ratings. Please see lines 74-76. 


\begin{changes}
The authors find that the “electoral systems employed work well in selecting the Condorcet winner most of the time” (p.7), which I find very interesting. Given the richness of the data, it seems like the authors could push their analysis further and check this propensity to pick the 	Condorcet winner varies across systems and voting rules. It could be interesting to discuss whether some systems are better at picking the Condorcet winner than others. This could lead to an interesting discussion of the paper’s policy implications.
\end{changes}

Thank you. We follow your suggestion and expand the discussion of the performance of Condorcet winners and losers in the revised manuscript. We include Table 2, which shows different aspects of the Condorcet efficiency across electoral systems and election types. It indeed reveals significant variation, that we hope is of interest to a broad audience and will stimulate much further research. 

\begin{changes}
I believe there in an inconsistency in the text as the abstract mentions 351 elections while Section 2 mentions only 221 elections.
\end{changes}

Thanks for noting. We removed that inconsistency. Note that the total number of elections has changed since we could use an updated version of the CSES dataset for our revision.


\newpage
\section{Concluding Remarks}

We would like to once again express our sincere gratitude for your exceptionally balanced and constructive remarks. We truly hope that the new version not only incorporates your valuable suggestions for improvement but also exhibits an overall increase in quality. We are eagerly looking forward to your feedback.

\bibliographystyle{elsarticle-harv}
\bibliography{CondorcetCycle}
\end{document}
